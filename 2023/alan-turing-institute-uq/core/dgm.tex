% \section{Continuous Diffusion Models}



% \begin{frame}{Motivating examples}
% Super-resolution

% \includegraphics[width=\linewidth]{images/superres.png}
% % \begin{enumerate}
% %     \item 
% % \end{enumerate}
    
% \end{frame}

% \begin{frame}{Motivating examples}
% \begin{enumerate}
%     \item[] Molecular conformation generation~\cite{xu2022GeoDiff}
%     \item[] Motif-Scaffolding~\cite{trippe2022Diffusion}
% \end{enumerate}

% \begin{center}
%     \vspace{-.5em}
%     % \includegraphics[angle=90,origin=c,width=0.5\linewidth]{images/protein_binding.png}
%     \includegraphics[width=\linewidth]{images/protein.png}
%     \vspace{.5em}
% \end{center}

% \end{frame}

% \begin{frame}{Motivating examples (Cont'd)}
% Probabilistic near future (nowcasting) prediction of precipitation~\cite{ravuri2021Skilful}
% \begin{center}
% \includegraphics[width=\linewidth]{images/rain_gen_models.png}
% % TODO(Vincent ): add back
% % \animategraphics[autoplay,loop,width=0.9\textwidth]{10}{images/rain/rain-}{0}{48}
% \end{center}
% \end{frame}



% % \includegraphics[width=\linewidth]{images/rain_gen_models.png}
% % % \begin{enumerate}
% % %     \item 
% % % \end{enumerate}
    
% % \end{frame}


% \begin{frame}{What is generative modelling?}

% Given $x_1, x_2, \ldots, x_n \sim p(x)$

% How to model the (unknown) density $p(x)$ and sample from it?

% \includegraphics[width=\linewidth]{images/generative_modelling_openai.png}

% \end{frame}

% % \begin{frame}{What is generative modelling?}
% %     Slide in the style of Valentin, although EBMs do not fit this!
% % \end{frame}

% % \begin{frame}{Motivating examples}
% % Data Augmentation

% % \centering \includegraphics[width=0.8\linewidth]{images/data_augmentation_gen.png}
% % % \begin{enumerate}
% % %     \item 
% % % \end{enumerate}
    
% % \end{frame}

% % \begin{frame}{What is generative modelling?}
% % % Generically, it is the parametrisation of a density. 
% % % We may have samples, or an normalised likelihood.
% % % We assume access to samples for training purposes (vs. unnormalised density for \emph{sampling}).
% % % We might want to draw more samples, or evaluate the likelihood.
% % \begin{itemize} \setbeamertemplate{itemize items}[triangle]
% %     \item Given a dataset of samples
% %     \item Obtain new samples
% %     \item Evaluate likelihood of samples
% % \end{itemize}
% % \vspace{1em}
% % \begin{tikzpicture}
% %     \node[inner sep=0pt] (ref) at (-5,0)
% %         {\includegraphics[width=0.25\textwidth]{images/ref_dist.png}};
% %     \node[inner sep=0pt] (gears) at (0,0)
% %         {\includegraphics[width=0.2\textwidth]{images/gears.jpg}};
% %     \node[inner sep=0pt] (map) at (5,0)
% %         {\includegraphics[width=0.25\textwidth]{images/map_dist.png}};
        
% %     \draw[->, ultra thick] (ref.east) -- (gears.west);
% %     \draw[->, ultra thick] (gears.east) -- (map.west);
    
% %     \node[above=0cm] at (ref.north) {\textbf{Simple distribution}};
% %     \node[above=0cm] at (map.north) {\textbf{Unknown complex distribution}};
% %     \node[below=0cm] at (ref.south) {\textbf{We can sample this}};
% %     \node[below=0cm] at (map.south) {\textbf{We have samples from this}};
% %     \node[above=0cm] at (gears.north) {\textbf{Some transformation}};
% % \end{tikzpicture}
% % Note fitting a Gaussian distribution to samples is \textit{technically} generative modelling, albeit a very simple version.
% % \end{frame}

% \begin{frame}{Deep generative models}
    
%     \begin{center}
%         \vspace{-.5em}
%         \includegraphics[width=0.9\linewidth]{images/dgms.png}
%         \captionof{figure}{\cite{albergo2022Building}}
%         \vspace{.5em}
%     \end{center}
   
% \end{frame}
    


% \begin{frame}{A narrative of generative model development}
% \begin{frame}{Deep generative models}

%     There are a number of existing generative model types:
    
%     \begin{columns}[t]
%     \begin{column}[t]{0.5\textwidth}
    
%     \begin{center}
%         \textbf{Likelihood based models}
%     \end{center}
%     \begin{itemize}
%         \item VAEs
%         \item Normalizing flows
%         \item Autoregressive models
%     \end{itemize}
    
%     \end{column}
%     \begin{column}[t]{0.5\textwidth}
    
%     \begin{center}
%         \textbf{Implicit models}
%     \end{center}
%     \begin{itemize}
%         \item Energy based models
%         \item GANs
%     \end{itemize}
    
%     \end{column}
%     \end{columns}
%     \vspace{1.5em}
%     \begin{columns}[t]
%     \begin{column}[t]{0.5\textwidth}
%     These models have quite restricted forms, and training via ELBOs can have difficulties.
%     \end{column}
%     \begin{column}[t]{0.5\textwidth}
%     % The adversarial losses of these models can be very tricky to train, and we have no access to likelihoods from the models.
%     Cannot evaluate likelihood.
%     GANs are tricky to train, whilst EBMs are slow to train as they require MCMC.
%     \end{column}
%     \end{columns}
% \end{frame}
    

% \begin{frame}{A narrative of generative model development}

% \begin{frame}{Deep generative models}
% \begin{table}[]
%     \centering
%     \begin{tabular}{lccc}
%     \toprule
%     & Explicit & Semi-implicit & Implicit \\
%     \midrule
%     Density & $p_{\theta}(x)$ & $p_{\theta}(x, z) = p_{\theta}(x|z)p_{\theta}(z)$ & No access \\
%     \midrule
%     Sampling & Efficient & Efficient & Efficient \\
%     %\midrule
%     %Reversibility & Yes & Approximate & No \\
%     \midrule 
%     Flexibility & Restricted & Less restricted & Flexible \\
%     \midrule
%     Example & ?? & ?? & ?? \\
%     \bottomrule
%     \end{tabular}%
%     \label{tab:genmodels}
% \end{table}
% \end{frame}

% \begin{frame}{Likelihood-based models}
% \vspace{1cm}
% \includegraphics[width=\linewidth]{images/likelihood_based_models.png}
% \begin{itemize}
%     \item Restricted
%     \item Doesn't reach state-of-the-art quality
% \end{itemize}
% \end{frame}

% \begin{frame}{Implicit models}
% \centering
% \includegraphics[width=0.8\linewidth]{images/implicit_models.png}
% \begin{itemize}
%     \item SOTA results
%     \item Flexible
%     \item Hard to train
%     \item Mode collapse is a problem
% \end{itemize}
% \end{frame}
\section{Continuous diffusion models}

% \begin{frame}{Principles of Score Generative Models (SGMs)}
\begin{frame}{Principles of continuous diffusion models}
\begin{center}
    \includegraphics[width=\textwidth]{images/sde.png}
    % TODO(Vincent): add
    % \animategraphics[autoplay,loop,width=0.85\textwidth]{10}{images/perturb_vp/perturb_vp-}{0}{99}
    \captionof{figure}{\cite{song2021Scorebased}}
\end{center}
\vspace{-0.5em}
% So far we have talked about discrete noise scales. Why? The discrete steps leave approximation inaccuracies, and tricky maths!
% Why going to the continuous setting? 1/ shed a new light on discrete SGMs 2/ easier quantitative bounds 3/ likelihood evaluation.
\begin{itemize} \setbeamertemplate{itemize items}[triangle]
    % \item Idea: Use a \textit{continuous} series of noise scales!
    \item Idea: Destruct data with \textit{continuous} series of noise.
    \item Do this by constructing an \textbf{SDE} forward noising process $(\fwd_t)_{t \in \ccint{0,T}}$.
    \item Have this noising converge to a \textbf{known distribution}.
    \item \textbf{Invert} this SDE noising process to get $(\bwd_t)_{t\in[0,T]} = (\fwd_{T-t})_{t\in[0,T]}$.
\end{itemize}
%
\end{frame}



\begin{frame}{Continuous noising processes}
%   \todo[inline]{introduce forward noising process}
%
The \textbf{Forward process} progressively perturbs the data following a SDE
\begin{equation}
  \label{eq:sde}
    \rmd \fwd_t = \hlblue{b(t, \fwd_t)} \rmd t + \hlred{\sigma(t, \fwd_t)} \rmd \bfB_t
\end{equation}
characterised by a drift $b$ and difffusion $\sigma$. $\rmd \bfB_t$ is Brownian
motion (think of it conceptually as $\rmd \bfB_t / \rmd t \sim \mathcal{N}(0, \rmd t)$.
\\
\pause
\vspace{0.5cm}
\textbf{Euler–Maruyama} discretisation with time step $\Delta_T \ll 1$ yields a Markov kernel: 
\begin{equation*}
     p(\v{Y}_{n+1} | \v{Y}_{n}) \approx \mathcal{N}(\v{Y}_{n+1}|\v{Y}_{n} +
     \Delta_T  \hlblue{b(t_n, \v{Y}_{n})}, { \Delta_T \hlred{\sigma^2(t_n,
     \v{Y}_n)}} \m{I}).
\end{equation*}
where $t_n = n \Delta T$.
\end{frame}

%
% \textbf{Langevin dynamics}:
% \begin{equation}
%   \label{eq:langevin}
% %  \rmd \fwd_t = b(t, \fwd_t) \rmd t + \sigma(t, \fwd_t) \sigma^\top (t, \fwd_t) \rmd \bfB_t^\M = - \tfrac{1}{2}~\nabla_{\fwd_t} U(\fwd_t) \rmd t + \rmd \bfB_t^\M,
%  \rmd \fwd_t = \hlblue{-\nabla_{\fwd_t} U(\fwd_t)} \rmd t + \hlred{\sqrt{2}} \rmd \bfB_t,
% \end{equation}
% admits \textbf{invariant} density: $\rmd \piinv/ \rmd \textrm{Leb}(x) \propto \rme^{-U(x)}$ \parencite[Section 2.4]{durmus2016high}.
%
% \begin{itemize} \setbeamertemplate{itemize items}[triangle]
%     \item \textbf{Brownian motion}: $\hlblue{b(t, \fwd_t) = 0} \Rightarrow \rmd \fwd_t = \hlred{\sqrt{2}} \rmd \bfB_t$
% \end{itemize}
% % \begin{equation}
% %   \label{eq:brownian}
% % \rmd \fwd_t = \hlred{\sqrt{2}} \rmd \bfB_t,
% % \end{equation}
% admits \textbf{invariant} measure $\mathrm{P}_{\text{ref}} = \textrm{Leb}$.
% \pause
% \vspace{1em}
% \begin{itemize} \setbeamertemplate{itemize items}[triangle]
%     \item \textbf{Ornstein-Uhlenbeck process}: $\hlblue{b(t, \fwd_t) = -\fwd_t} \Rightarrow \rmd \fwd_t = \hlblue{- \fwd_t} \rmd t + \hlred{\sqrt{2}} \rmd \bfB_t$
% \end{itemize}
% % \begin{equation}
% %   \label{eq:langevin}
% % \rmd \fwd_t = \hlblue{- \fwd_t} \rmd t + \hlred{\sqrt{2}} \rmd \bfB_t,
% % \end{equation}
% admits \textbf{invariant} density: $\rmd \mathrm{P}_{\text{ref}} / \rmd \textrm{Leb} = \c{N}(0, \Id)$.
% \pause
% %
% \begin{itemize} \setbeamertemplate{itemize items}[triangle]
%     \item \textbf{Discretisation} yields Markov kernel: 
% \end{itemize}
% $\rmd \fwd_t = \hlblue  {b(\fwd_t)} \rmd t + \hlred{\sqrt{2}} \rmd \bfB_t \Rightarrow p_{t+1|t}(x_{t+1}|x_{t}) = \mathcal{N}(x_{t+1}|x_{t} + \gamma  \hlblue{b(x_{t})}, {\hlred{\sqrt{2}}^2 \gamma} \Id)$.
% %
% \begin{itemize} \setbeamertemplate{itemize items}[triangle]
%     \item \textbf{Brownian motion}: $\hlblue{b(t, \fwd_t) = 0} \Rightarrow \rmd \fwd_t = \hlred{\sqrt{2}} \rmd \bfB_t$
% \end{itemize}
% % \begin{equation}
% %   \label{eq:brownian}
% % \rmd \fwd_t = \hlred{\sqrt{2}} \rmd \bfB_t,
% % \end{equation}
% admits \textbf{invariant} measure $\mathrm{P}_{\text{ref}} = \textrm{Leb}$.
% \pause
% \vspace{1em}
% \begin{itemize} \setbeamertemplate{itemize items}[triangle]
%     \item \textbf{Ornstein-Uhlenbeck process}: $\hlblue{b(t, \fwd_t) = -\fwd_t} \Rightarrow \rmd \fwd_t = \hlblue{- \fwd_t} \rmd t + \hlred{\sqrt{2}} \rmd \bfB_t$
% \end{itemize}
% % \begin{equation}
% %   \label{eq:langevin}
% % \rmd \fwd_t = \hlblue{- \fwd_t} \rmd t + \hlred{\sqrt{2}} \rmd \bfB_t,
% % \end{equation}
% admits \textbf{invariant} density: $\rmd \mathrm{P}_{\text{ref}} / \rmd \textrm{Leb} = \c{N}(0, \Id)$.

% \pause
% \textbf{Euler–Maruyama} discretisation with time step $\gamma_t$ yields a Markov kernel: 
% \begin{equation*}
%      p(\v{y}_t | \v{y}_{t-1}) \approx \mathcal{N}(\v{y}_{t}|\v{y}_{t-1} + \gamma_t  \hlblue{b(t, \v{y}_{t-1})}, { \gamma_t \hlred{\sigma(t, \v{y}_t)}^2} \m{I}).
% \end{equation*}



\begin{frame}{Example: Ornstein–Uhlenbeck process on $\mathbb{R}^2$}

Let the data $\fwd_0 \in \mathbb{R}^2$ be distributed according to a
\emph{known} 2D Gaussian with a correlation coefficient $\rho \approx 1$.

% We specify the drift as $b(t, \fwd_t) = - \fwd_t$ and diffusion $\sigma(t,\fwd_t) = \sqrt{t}$ to get a 
We specify the drift to be linear and the diffusion coefficient to be constant
\begin{equation}
  \label{eq:example:sde}
    \rmd \fwd_t = \hlblue{- \fwd_t} \rmd t + \hlred{\sqrt{2}} \rmd \bfB_t.
\end{equation}

\begin{figure}
\centering
% \includegraphics[width=\textwidth]{images/example_2d/forward.png}
\includegraphics[width=\linewidth,trim={5cm 0 5cm 0},clip]{images/example_2d/forward.pdf}
\caption{Forward OU process on 2D data.}
\end{figure}
\pause
\vspace{-5mm}
% At $t=T$ we end up with a distribution approximately $\mathcal{N}(\v{0},\v{I})$.
\end{frame}

% \begin{frame}{Continuous score-based models}
% \begin{frame}{Continuous score-based models: Taking the limit $\gamma \rightarrow 0$}
% \begin{frame}{Taking the limit $\gamma \rightarrow 0$~\cite{song2021Scorebased}}



% \begin{frame}{Discrete diffusion models as discretised continuous models~\cite{song2021Scorebased}}
% %
% % $p_{t+1|t}(x_{t+1}|x_{t}) = \mathcal{N}(x_{t+1}|x_{t} + \gamma  \hlblue{b(x_{t})}, {\hlred{\sqrt{2}}^2 \gamma} \Id) \Rightarrow_{\gamma \rightarrow 0} \rmd \fwd_t = \hlblue{b(\fwd_t)} \rmd t + \hlred{\sqrt{2}} \rmd \bfB_t$.
% %
% These things look oddly familiar!

% \begin{table}
%     \centering
%     \renewcommand*{\arraystretch}{1.5}
%     \begin{tabular}{c|cc}
%         & SMLD \cite{song2019generative}  & DDPM \cite{ho2020denoising} \\
%         \hline
%         $p(\v{y}_{t} | \v{y}_{t-1})$ & $\c{N}\del{\v{y}_{t-1}, \sigma_t^2 - \sigma_{t-1}^2}$ & $\c{N}\del{(1- \gamma_t) \v{y}_{t-1}, 2\gamma_t \m{I}}$ \\
%         Cont. $\hlblue{b(t, \fwd_t)}$ & $\hlblue{0}$ & $\hlblue{-\fwd_t}$ \\
%         Cont. $\hlred{\sigma(t, \fwd_t)}$ & $\hlred{\sqrt{\frac{\rmd\sigma^2(t)}{\rmd t}}}$ & $\hlred{\sqrt{2}}$ \\
%         $p(\v{y}_T)$, $T$ is 'big' & $\c{N}\del{0, \int_0^T \sigma^2(t) \rmd t \m{I}}$ & $\c{N}(0, \m{I})$
%     \end{tabular}
% \end{table}
% \end{frame}

% \vspace{-0.3em}
% \begin{itemize} \setbeamertemplate{itemize items}[triangle]
%     \item \textbf{SMLD} \cite{song2019generative,song2020improved}:
% \end{itemize}
% \begin{itemize}
%     \item Kernel: $p_{t+1|t}(x_{t+1}|x_{t}) = \mathcal{N}(x_{t}, {2 \gamma})$.
%     \item Update: $\fwd_{t+1} = \fwd_t + \sqrt{\gamma} \hlred{\sqrt{2}} \bfZ_{t+1}$ with $\bfZ_{t+1} \sim \mathcal{N}(0,\Id)$.
%     \item SDE: \textbf{Brownian motion}: $\rmd \fwd_t = \hlred{\sqrt{2}} \rmd \bfB_t,\quad \fwd_0 \sim p_0 $.
%     \item Conditional: $\fwd_t|\fwd_0 = \fwd_0 + \bfB_t$.
%     % \item Invariant: $\mathrm{P}_{\text{ref}} = \text{Leb}$.
%     \item Prior: $\piinv     = \c{N}(0, 2T \Id)$.
%     \end{itemize}
% \pause
% %
% \begin{itemize} \setbeamertemplate{itemize items}[triangle]
%     \item \textbf{DDPM} \cite{sohl2015deep,ho2020denoising}:
% \end{itemize}
% \begin{itemize}
%         \item kernel: $p_{t+1|t}(x_{t+1}|x_{t}) = \c{N}(x_{t+1}|(1-\gamma)x_{t},2\gamma)$.
%         \item Update:  $\fwd_{t+1} = \fwd_{t} \gamma \hlblue{- \fwd_{t}} + \sqrt{\gamma} \hlred{\sqrt{2}} \bfZ_{t+1}$ with $\bfZ_{t+1} \sim \c{N}(0, \Id)$
%         \item SDE: \textbf{Ornstein-Uhlenbeck} process: $\rmd \fwd_t = \hlblue{- \fwd_t} \rmd t + \hlred{\sqrt{2}} \rmd \bfB_t,\quad \fwd_0 \sim p_0 $.
%         \item Conditional: $\fwd_t|\fwd_0 = e^{-t}\fwd_0 + \bfB_{1-e^{-2t}}$.
%         % \item Invariant: $\rmd \mathrm{P}_{\text{ref}} / \rmd \text{Leb} \triangleq \piinv = \c{N}(0, \Id)$.
%         \item Prior: $\piinv = \c{N}(0, \Id)$.

%     \end{itemize}




% \begin{frame}{Stochastic differential equations}
\begin{frame}{Continuous score-based models: Time reversal process}
    % \citeauthor{debortoli2021diffusion}
\vspace{-0.2em}
% Taking the limit $\gamma \rightarrow 0$...
    
% \begin{equation}\label{eq:forward_SDE}
%   \rmd \fwd_t =\hlblue{-\fwd_t} \rmd t + \hlred{\sqrt{2}} \rmd \bfB_t,\quad \fwd_0 \sim p_0 ,
% \end{equation}
% \begin{equation}\label{eq:marginal_sde}
%   \fwd_t|\fwd_0 = e^{-1}\fwd_0 + \bfB_{1-e^{-2t}}
% \end{equation}
%
% \begin{theorem}{Time-reversal process~\cite{cattiaux2021time,haussmann1986time}}{}
\vspace{5mm}
\begin{theorem}{\cite{cattiaux2021time,haussmann1986time}}{}
% Under mild conditions on $p_0$,
The time-reversed process
$(\bwd_t)_{t \geq 0} = (\fwd_{T-t})_{t \in \ccint{0,T}}$, 
with forward process $\rmd \fwd_t = \hlblue{b(t, \fwd_t)} \rmd t + \hlred{\sigma(t)} \rmd \bfB_t$,
also satisfies an SDE given by
\begin{equation*}
\label{eq:backward_SDE}
%   \rmd \bwd_t = \hlblue{\left[  \bwd_t + 2 \nabla \log p_{T-t}(\bwd_t)\right] } \rmd t + \hlred{\sqrt{2}} \rmd \bfB_t,\quad \bwd_0 \sim p_T.
  \rmd \bwd_t = \left[ -\hlblue{b(T-t,\bwd_t)} + \hlred{\sigma(T-t)}^2 \hlyellow{\nabla \log p_{T-t}(\bwd_t)}\right] \rmd t + \hlred{\sigma(T-t)} \rmd \bfB_t,
\end{equation*}
assuming $ \bwd_0$ is distributed the same as $\fwd_T$.
\end{theorem}
%
\pause
%
Challenges:
\vspace{-0.5em}
\begin{enumerate}
    \item We do not have access to $\fwd_T \Rightarrow$ Approximate as $\fwd_T \approx \fwd_\infty$!
    \item The Stein score $\hlyellow{\nabla \log p_t}$ is intractable (requires solving Fokker-Planck...) $\Rightarrow$ learn it!
    \item Cannot solve the SDE exactly $\Rightarrow$ discretise!
\end{enumerate}
%

% \vspace{1.em}
% %
% Training the score network $\hlyellow{\mathbf{s}_\theta}: \R_+ \times \rset^d \rightarrow \rset^d$ is the same as in the discrete setting, with the (denoising) score matching loss, but with continuous time.
% \begin{equation*}
%     % \ell_t(\mathbf{s}) = \PE_{x_0,t,x_t}{\left[ \lambda(t) \|\mathbf{s}(x_t) - \nabla_{x_t} \log p_{t|0}(x_t|x_0)\|^2 \right]}
%     \ell_{\mathrm{dsm}, \lambda}(\hlyellow{\mathbf{s}_\theta}) = \E_{p(t) p(\fwd_0) p(\fwd_t | \fwd_0)}{\left[ \lambda(t) \norm{\hlyellow{\mathbf{s}_\theta}(t, \fwd_t) - \nabla_{\fwd_t} \log p_{t|0}(\fwd_t|\fwd_0)}^2 \right]}.
% \end{equation*}
% with $p(x_0,t,x_t) = p_0(x_0) p(t) p_{t|0}(x_t|x_0)$, $p(t) =\c{U}(0, T)$.
\end{frame}


% Sampling with \textbf{Euler-Maruyama}
% \begin{equation*}
% % \bwd_{t+\gamma} = \gamma \left[\bwd_{t} + 2 \nabla \log p_{T-t}(\bwd_t) \right] + \sqrt{2 \gamma} \bfZ_{t}, \quad \bfZ_{t} \sim \c{N}(0, \Id).
% \bwd_{t+\gamma} = \gamma \left[-b(T-t,\bwd_t) + \sigma^2(T-t) \nabla \log p_{T-t}(\bwd_t) \right] + \sqrt{\gamma} \sigma(T-t) \bfZ_{t}, \quad \bfZ_{t} \sim \c{N}(0, \Id).
% \end{equation*}
% \pause


\begin{frame}{Learning the score \cite{hyvarinen2005estimation,vincent2011connection,song2021Scorebased}}

\begin{itemize}
    \item The Stein score $\hlyellow{\nabla \log p_t} =\nabla \log \int p_{data}(\fwd_0) p_{t\mid 0}(\fwd_t \mid \fwd_0) \rmd \fwd_0$ is intractable.
    \item However, it can be shown that the score is the minimiser of regression objective
    \begin{equation}
        \label{eq:scoreobjective}
    {\nabla_{\fwd_t} \log p_t(\fwd_t)} = \argmin_{\v{s} \in \mathcal{S}} \expeLigne{\|{\mathbf{s}(t, \fwd_t)} - \nabla_{\fwd_t} \log p_{t|0}(\fwd_t|\fwd_0)\|^2},
    \end{equation}
    where the expectation is taken over the joint $(t, \fwd_0, \fwd_t)$.
% \begin{itemize}
%     \item Score matching identity \cite{vincent2011connection,hyvarinen2005estimation}
%     \begin{equation}\label{eq:scoreidentity}
%         % \textstyle
%         \hlyellow{\nabla_{\fwd_t} \log p_t(\fwd_t)} =
%         \mathbb{E}_{p_{0\mid t}(\fwd_0 \mid \fwd_t)}\left[\nabla_{\fwd_t} \log p_{t\mid 0}(\fwd_t \mid \fwd_0)\right].
%       \end{equation}
%       where $p_{0\mid t}(\fwd_0 \mid \fwd_t)$ is the reverse and $p_{t|0}(\fwd_t|\fwd_0)$ is the forward transition density.

%     \item By definition of the conditional expectaton, it follows that the score is the minimiser of
%     \begin{equation}
%         \label{eq:scoreobjective}
%     {\nabla_{\fwd_t} \log p_t(\fwd_t)} = \argmin_{\v{s} \in \mathcal{S}} \expeLigne{\|{\mathbf{s}(t, \fwd_t)} - \nabla_{\fwd_t} \log p_{t|0}(\fwd_t|\fwd_0)\|^2},
%     \end{equation}
%     where the expectation is taken over the joint $(t, \fwd_0, \fwd_t)$.
%     \item We parameterise the class of functions $\mathbf{s}: \ccint{0,T} \times \rset^d \to \rset^d$ by a neural network $\v{s}_\theta$.
% \end{itemize}
    \item We have access to the conditional forward density $p_{t\mid 0}$ in closed form for OU processes.
    \item This readily gives a loss to train a neural network $\hlyellow{\mathbf{s}_\theta: [0, T] \times \rset^d \rightarrow \rset^d}$ parameterisation of the score
\begin{equation}
    \textstyle
    \mathcal{L}(\theta)
     = \mathbb{E} [\lambda(t) \|  \hlyellow{\mathrm{s}_\theta(t, \fwd_t)} - \nabla \log p_{t}({\fwd}_t| \fwd_0)\|^2].
\end{equation}
\end{itemize}
\end{frame}


% \begin{frame}{Score approximation}

%     \begin{itemize} \setbeamertemplate{itemize items}[triangle]
%         % \item The Stein score $\nabla \log p_t$ is hard to compute (requires solving Fokker-Planck!)
%         \item One make use of the following denoising score matching identity
%         \begin{equation}\label{eq:scoreidentity}
%             \textstyle
%             \hlyellow{\nabla_{y_t} \log p_t(y_t)} = \int_{\rset^d} \nabla \log p_{t|0}(y_t|y_0)~p_{0|t}(y_0|y_t) \rmd y_0,
%           \end{equation}
%           where $p_{t|0}(y_t|y_0)$ is the transition density of the forward noising process.
%         \item It follows directly that $\hlyellow{\nabla \log p_t}$ is the minimiser of $\mathcal{L}(\mathbf{s}) = \expeLigne{\|\hlyellow{\mathbf{s}(t, \fwd_t)} - \nabla_{y_t}
%         \log p_{t|0}(\fwd_t|\fwd_0)\|^2}$ over functions $\hlyellow{\mathbf{s}: \ccint{0,T} \times \rset^d \to \rset^d}$.
%         \item This readily gives a loss over the finite parameters $\theta \in \Theta$ of the neural network $\hlyellow{\mathbf{s}_\theta: [0, T] \times \rset^d \rightarrow \rset^d}$:
%     %     \begin{equation*}
%     %         % \ell_t(\mathbf{s}) = \PE_{x_0,t,x_t}{\left[ \lambda(t) \|\mathbf{s}(x_t) - \nabla_{x_t} \log p_{t|0}(x_t|x_0)\|^2 \right]}
%     %     \ell_{\mathrm{dsm}, \lambda}(\hlyellow{\mathbf{s}_\theta}) = \E_{p(t) p(\fwd_0) p(\fwd_t | \fwd_0)}{\left[ \lambda(t) \norm{\hlyellow{\mathbf{s}_\theta}(t, \fwd_t) - \nabla_{\fwd_t} \log p_{t|0}(\fwd_t|\fwd_0)}^2 \right]}.
%     % \end{equation*}
%     \begin{align}
%             % \label{eq:loss}
%         \textstyle
%         \mathcal{L}(\theta; \lambda(t))
%          = \mathbb{E} [\lambda(t) \|  \hlyellow{\mathrm{s}_\theta(t, \fwd_t)} - \nabla \log p_{t}({\fwd}_t| \fwd_0)\|^2]
%     \end{align}
%     where the expectation is taken over the joint $(t, \fwd_0, \fwd_t)$.
%     \end{itemize}
%     \end{frame}



\begin{frame}{Sampling from the reverse process in practice}
% As previously mentioned, 
The (true) reverse process is given by
\begin{equation*}
  \rmd \bwd_t = \left[ -\hlblue{b(T-t,\bwd_t)} + \hlred{\sigma(T-t)}^2 \hlyellow{\nabla \log p_{T-t}(\bwd_t)}\right] \rmd t + \hlred{\sigma(T-t)} \rmd \bfB_t.
\end{equation*}
We make the following approximations
\begin{enumerate}
    \item Initialise $\bwd_0 \sim \mathcal{N}(\v{0},\v{I})$ which approximates $p(\fwd_{T})$.
    \item Use score approximation $\hlyellow{\nabla \log p_{T-t}(\bwd_t)} = \v{s}_\theta(T - t, \bwd_t)$.
    \item Discretise SDE using Euler–Maruyama method
\end{enumerate}
\begin{figure}
\centering
\includegraphics[width=\linewidth,trim={5cm 0 5cm 0},clip]{images/example_2d/reverse.pdf}
% \caption{Reverse process}
\end{figure}
% \begin{equation*}
%      % \fwd_t \approx \fwd_{t-1} + \gamma_t \hlblue{b(t, \fwd_{t-1})} + \sqrt{\gamma_t} \hlred{\sigma(t, \fwd_t)} \bfZ_t , \quad \bfZ_t \sim \c{N}(\v{0},\m{I})
%      \hspace{-0.3cm}\bwd_{n+1} \approx \bwd_{n} + \Delta_T [\hlblue{-b(T - t_n, \bwd_{n})} +
%      \hlred{\sigma(T - t_n)}^2  \hlyellow{\v{s}_\theta(T-t_n,\bwd_{n})} + \sqrt{\Delta_T} \hlred{\sigma(T -
%      t_n, \bwd_n)} \bfZ_n.% , \quad \bfZ_t \sim \c{N}(\v{0},\m{I})
% \end{equation*}
% \pause
% This discretisation however leads to some error in the sampling from the SDE. We can \textbf{correct} this error by running Langevin dynamics after this step targeting the distribution $\c{L}(\fwd_t)$. We run these dynamics along another time axis $s$ with the (time-reversal) SDE
% \begin{equation*}
%     % \rmd\fwd_t^s = -\nabla \log p(\fwd_t) \rmd s + \sqrt{2} \rmd\bfB_s
%     \rmd\bwd_t^s = \hlyellow{\nabla \log p(\bwd_t^s)} \rmd s + \sqrt{2} \rmd\bfB_s
% \end{equation*}
% which can be discretised as
% \begin{equation*}
%     % \fwd_t^{s} \approx \fwd_t^{s-1} - \gamma_{t,s}\nabla \log p(\fwd_t^{s-1}) + \sqrt{2\gamma_{t,s}} \bfZ_t^s , \quad \bfZ_t^s \sim \c{N}(\v{0},\m{I}).
%     \bwd_t^{s} \approx \bwd_t^{s-1} + \gamma_{t,s} \hlyellow{\nabla \log p(\bwd_t^{s-1})} + \sqrt{2\gamma_{t,s}} \bfZ_t^s , \quad \bfZ_t^s \sim \c{N}(\v{0},\m{I}).
% \end{equation*}
% We can anneal $\gamma_{t,s}$ to get close to $p(\fwd_t)$.
% \begin{center}
    % \includegraphics[width=\textwidth]{images/sde.png}
    % TODO(Vincent): add
    % \animategraphics[autoplay,loop,width=0.7\textwidth]{10}{images/denoise_vp/denoise_vp-}{0}{99}
% \end{center}
\end{frame}

\begin{frame}{Improved sampling using Langevin dynamics}
\begin{itemize}
    \item Euler-Maruyama method introduces discretisation errors.
    \item Song et al. 2021 suggest to use Langevin dymanics to correct each reverse step.
\end{itemize}
\pause
\textbf{Langevin dynamics}:
\begin{equation}
  \label{eq:langevin}
%  \rmd \fwd_t = b(t, \fwd_t) \rmd t + \sigma(t, \fwd_t) \sigma^\top (t, \fwd_t) \rmd \bfB_t^\M = - \tfrac{1}{2}~\nabla_{\fwd_t} U(\fwd_t) \rmd t + \rmd \bfB_t^\M,
 \rmd \fwd_t = \hlblue{\nabla_{\fwd_t} \log p(\fwd_t)} \rmd t + \hlred{\sqrt{2}} \rmd \bfB_t,
\end{equation}
As \( t \to \infty \), the dynamics converges towards the distribution $p(\cdot )$.
\pause

\textbf{Predictor-Corrector sampling}
\vspace{-1.8cm}
\begin{figure}
    \centering
    \begin{center}
    \resizebox{.6\linewidth}{!}{
    \begin{tikzpicture}
    \Large
        \draw [stealth-, line width = .05cm] (0,0) .. controls  (2,-4) and (3.,2) .. (6,0);
        \draw [-stealth, line width = .05cm, C0, dashed] (0,0) -- (1.3,-2.3);
        \draw [-stealth, line width = .05cm, C3, dashed] (1.3,-2.3) -- (1.3,-1.9);
        \draw [-stealth, line width = .05cm, C3, dashed] (1.3,-1.9) -- (1.3,-1.7);
        \draw [-stealth, line width = .05cm, C3, dashed] (1.3,-1.7) -- (1.3,-1.4);
        \draw [-stealth, line width = .05cm, C0, dashed] (1.3,-1.45) -- (2.8,-1.35);
        \draw [-stealth, line width = .05cm, C3, dashed] (2.8,-1.35) -- (2.4, -0.9);
    %   \draw [-stealth, line width = .05cm, blue, dashed] (0,0) -- (1.3,-2.3);
    %   \draw [-stealth, line width = .05cm, red, dashed] (1.3,-2.3) -- (1.3,-1.2);
    %   \draw [-stealth, line width = .05cm, blue, dashed] (1.3,-1.2) -- (2.8,-0.7);
    %   \draw [-stealth, line width = .05cm, red, dashed] (2.8,-0.7) -- (2.6, -0.5);
        \node [below=0.2cm] at (6,0) {\footnotesize $p_0$};
        \node [left=0.0cm] at (0,0) {\footnotesize $p(\bwd_0) = p_T$};
        \node [left=0.2cm] at (1.3,-2.1) {\footnotesize $p(\bwd_\gamma^0)$};
        \node [right=0.1cm] at (1.3,-2.) {\footnotesize $p(\bwd_\gamma^1)$};
        \node [right=0.1cm] at (1.3,-1.65) {\footnotesize $\cdots$};
    %   \node [below right=0.1cm] at (1.3,-1.2) {$\mathcal{L}(\fwd_\gamma^1)$};
    %   \node [above=0.05cm] at (1.3,-1.2) {$\mathcal{L}(\fwd_\gamma^3)$};
        \node [above=0.05cm] at (1.3,-1.35) {\footnotesize $p_{\gamma}$};
    %   \node [right=0.2cm] at (2.8,-0.7) {$\mathcal{L}(\fwd_{2\gamma}^0)$};
        %   \node [above left=0.1cm] at (2.6, -0.5) {$\mathcal{L}(\fwd_{2\gamma}^1)$};
        \node [below right=0.1cm] at (2.8,-0.8) {\footnotesize $p(\bwd_{2\gamma}^0)$};
        \node [right=0.1cm] at (2.6, -0.7) {\footnotesize $p(\bwd_{2\gamma}^1)$};
        \node [above left=0.05cm] at (2.6, -0.9) {\footnotesize $p_{2\gamma}$};
    \end{tikzpicture}
    }
    \end{center}
\end{figure}
\vspace{-2.cm}
{\footnotesize Credits to Valentin De Bortoli for graphic.}
\end{frame}


% \begin{frame}{Predictor-Corrector sampling}
% \vspace{-7.em}
% \begin{center}
% \begin{tikzpicture}
%   \draw [stealth-, line width = .05cm] (0,0) .. controls  (2,-4) and (4,4) .. (6,0);
%   \draw [-stealth, line width = .05cm, solidblue, dashed] (0,0) -- (1.3,-2.3);
%   \draw [-stealth, line width = .05cm, solidred, dashed] (1.3,-2.3) -- (1.3,-1.9);
%   \draw [-stealth, line width = .05cm, solidred, dashed] (1.3,-1.9) -- (1.3,-1.5);
%   \draw [-stealth, line width = .05cm, solidred, dashed] (1.3,-1.5) -- (1.3,-1.2);
%   \draw [-stealth, line width = .05cm, solidblue, dashed] (1.3,-1.2) -- (2.8,-0.7);
%   \draw [-stealth, line width = .05cm, solidred, dashed] (2.8,-0.7) -- (2.6, -0.5);
% %   \draw [-stealth, line width = .05cm, blue, dashed] (0,0) -- (1.3,-2.3);
% %   \draw [-stealth, line width = .05cm, red, dashed] (1.3,-2.3) -- (1.3,-1.2);
% %   \draw [-stealth, line width = .05cm, blue, dashed] (1.3,-1.2) -- (2.8,-0.7);
% %   \draw [-stealth, line width = .05cm, red, dashed] (2.8,-0.7) -- (2.6, -0.5);
%   \node [below=0.2cm] at (6,0) {$p_{\textrm{data}}$};
%   \node [left=0.3cm] at (0,0) {$\mathcal{L}(\bwd_0) = p_T$};
%   \node [below=0.2cm] at (1.3,-2.3) {$\mathcal{L}(\bwd_\gamma^0)$};
%   \node [right=0.1cm] at (1.3,-1.9) {$\mathcal{L}(\bwd_\gamma^1)$};
%   \node [right=0.1cm] at (1.3,-1.5) {$\cdots$};
% %   \node [below right=0.1cm] at (1.3,-1.2) {$\mathcal{L}(\bwd_\gamma^1)$};
% %   \node [above=0.05cm] at (1.3,-1.2) {$\mathcal{L}(\bwd_\gamma^3)$};
%   \node [above=0.05cm] at (1.3,-1.2) {$p_{\gamma}$};
% %   \node [right=0.2cm] at (2.8,-0.7) {$\mathcal{L}(\bwd_{2\gamma}^0)$};
%     %   \node [above left=0.1cm] at (2.6, -0.5) {$\mathcal{L}(\bwd_{2\gamma}^1)$};
%   \node [below right=0.1cm] at (2.8,-0.7) {$\mathcal{L}(\bwd_{2\gamma}^0)$};
%   \node [right=0.1cm] at (2.6, -0.5) {$\mathcal{L}(\bwd_{2\gamma}^1)$};
%   \node [above left=0.05cm] at (2.6, -0.5) {$p_{2\gamma}$};
% \end{tikzpicture}
% \end{center}
% \vspace{-2.5em}
% % \caption{
%     % Illustration of the effect of the corrector step on SGM.
%     The black line corresponds to the dynamics of the noising process 
%   $(p_t)_{t \in \ccint{0,T}}$. The \textcolor{solidblue}{blue dashed lines} correspond to the
%   predictor step (going backward in time) and the \textcolor{solidred}{red dashed lines} correspond to
%   the corrector step (projecting back onto the forward dynamics).
% %   Note that
% %   $\mathcal{L}(X_1^\gamma) \approx p_{T- \gamma}$ and
% %   $\mathcal{L}(X_2^\gamma) \approx p_{T- 2\gamma}$.
% % }

% \end{frame}
  

% \begin{frame}{Predictor-Corrector sampling from the backwards process}
% %
% \begin{algorithm}[H]
% % \caption{\small GRW-c (Geodesic Random Walk with corrector)}
% \caption{\small Predictor-Corrector}
% \label{alg:grw-c}
% \begin{algorithmic}[1]
%  \small
% %   \Require $T, N, \gamma = T / N, X_0, b, \sigma, \mathrm{P}$
% %   \Require $T, N, \gamma = T / N, X_0, b, \beta_t$
%     \Require $\bwd_0$, $T$, $N$, $S$, $\gamma = T/N$, $\gamma_{s}$
% %   \State $\gamma = T / N$ \Comment Step-size
%   \For{\textcolor{solidblue}{$k \in \{0, \dots, N-1\}$}}
%   \State \textcolor{solidblue}{/// PREDICTOR STEP}
%   \State \textcolor{solidblue}{${\bfZ}_{k+1} \sim \mathrm{N}(0, \Id)$} \Comment Standard Gaussian noise% in ambient space $\rset^p$
% %   \State \textcolor{solidblue}{$Z_{k+1/2} = \mathrm{P}(X_k^\gamma) \bar{Z}_{k+1/2}$} \Comment Projection in the tangent space $\mathrm{T}_x \M$ 
% %   \State \textcolor{solidblue}{$W_{k+1} = \gamma b(k \gamma, X_k) + \sqrt{\gamma} \sigma(k \gamma, X_k) Z_{k+1}$} \Comment Euler--Maruyama step %on tangent space 
% %   \State \textcolor{solidblue}{$W_{k+1} = \gamma b(k \gamma, X_k) + \sqrt{\gamma} \sigma(k \gamma, X_k) Z_{k+1}$} \Comment Euler--Maruyama step %on tangent space 
%   \State \textcolor{solidblue}{$\bwd_{k+1} = \bwd_{k} + \gamma \sbr{-b(T-k \gamma, \bwd_k) + \nabla \log p_{T-k\gamma}(\bwd_{k})} + \sqrt{\gamma} \bfZ_{k+1}$} \Comment E-M step %on tangent space 
% %   \State \textcolor{solidblue}{$X_{k+1} = \exp_{X_k}[W_{k+1}]$} \Comment %Geodesic projection onto $\M$

%   \State \textcolor{solidred}{/// CORRECTOR STEP}
%   \State \textcolor{solidred}{${\bwd}_{k+1}^0 = {\bwd}_{k+1}$}
%   \For{\textcolor{solidred}{$s \in \{0, \dots, S-1\}$}}
%   \State \textcolor{solidred}{${\bfZ}_{k+1}^s \sim \mathrm{N}(0, \Id)$} \Comment Standard Gaussian noise% in ambient space $\rset^p$
% %   \State \textcolor{solidred}{$Z_{k+1} = \mathrm{P}(X_{k+1/2}^\gamma) \bar{Z}_{k+1}$} \Comment Projection in the tangent space $\mathrm{T}_x \M$ 
% %   \State \textcolor{solidred}{$W_{k+1}^{l+1} = \tfrac{\gamma}{2} b(k \gamma, X_{k}^l) + \sqrt{\gamma} \sigma(k \gamma, X_{k}^l) Z_{k+1}$} \Comment Euler--Maruyama step% on tangent space \label{line:dete}
%   \State \textcolor{solidred}{$\bwd_{k+1}^{s+1} = \bwd_{k+1}^{s} + \gamma_s \nabla \log p_{T-k\gamma}(\bwd_{k+1}^{s}) + \sqrt{2 \gamma_s} \bfZ_{k+1}^s$} \Comment Langevin step% on tangent space \label{line:dete}
% %   \State \textcolor{solidred}{$X_{k+1}^{l+1} = \exp_{X_{k}^l}[W_{k+1}^{l+1}]$} \Comment %Geodesic projection onto $\M$  
%   \EndFor
%   \State \textcolor{solidred}{${\bwd}_{k+1} = {\bwd}_{k+1}^S$}
%   \EndFor
%   \State {\bfseries return} $\bwd_N$
% \end{algorithmic}
% \end{algorithm}

    
% \end{frame}

% \begin{frame}{Important 'tricks' - The SDE}
    
% % \todo[inline]{Also: Loss function weighting, corrector and more}

% \textbf{Noise scheduling}:
% For whatever SDE we are looking at, remap
% \begin{itemize}
%     \item $\hlblue{b(t, \fwd_{t})} \to \beta(t) \hlblue{b(t, \fwd_{t})}$
%     \item $\hlred{\sigma(t)} \to \sqrt{\beta(t)} \hlred{\sigma(t)}$
% \end{itemize}
% Doing this is equivalent to \textit{re-scaling time} such that $t \mapsto \int_0^t \beta(s) ds$. We can set $\beta(t)$ to reduce the step size where the score norm is high.

% % The aim is to have a linear convergence $\mathbb{W}(\mathcal{L}(\fwd_t), \piinv) \approx \mathbb{W}(\mathcal{L}(\fwd_t), p_0) * (1 - t)_+$ of the discrete steps from reference measure to data.

% \pause

% \textbf{Process truncation}: As $t \to 0$, the score blows up, i.e.\ $\|\hlyellow{\nabla \log p(x_t)}\| \rightarrow \infty$, with the manifold hypothesis~\cite{debortoli2022Convergence}, or we only have access to finite data.
% % or since finite data gives a product of Dirac masses distribution.


% Truncating the process so that $t \in \sbr{\epsilon, T}$ prevents this issue.
% % This is equivalent to adding a small amount of the noise to the data, and puts density everywhere.
% This is equivalent to smoothing the data with some small noise, effectively extending the support of the data distribution to $\rset^d$.
% % adding a small amount of the noise to the data, and puts density everywhere.


% % \begin{itemize} \setbeamertemplate{itemize items}[triangle]
% %     \item \textbf{Noise scheduling} $\beta(t)$
% %     \begin{itemize} \setbeamertemplate{itemize items}[circle]
% %     \item $\rmd \fwd_t = \hlblue{-\beta(t)~\fwd_t} \rmd t + \hlred{\sqrt{2 \beta(t)}} \rmd \bfB_t$.
% %     \item `Rescale time': $t \mapsto \int_0^t \beta(s) ds$.
% %     \item Spend more time when the score has high norm, i.e.\ when $t$ is small.
% %     \item Aim: linear rate of convergence, i.e.\ $\mathbb{W}(\mathcal{L}(\fwd_t), \piinv) \approx \mathbb{W}(\mathcal{L}(\fwd_t), p_0) * (1 - t)$.
% %     % \item Aim: $\mathbb{W}(\mathcal{L}(\fwd_t), \piinv)$ roughly linear with $t$ and $\approx 0$ when $t=1$.
% %     \end{itemize} \setbeamertemplate{itemize items}[triangle]
% %     \item Truncate processes between $[\epsilon, T]$ with $\epsilon > 0$ since $\|\nabla \log p(x_t)\| \rightarrow \infty$ as $t \rightarrow 0$ with manifold hypothesis~\cite{debortoli2022Convergence}.
% %     \item \textbf{Loss weighting} $\lambda(t)$
% %     \begin{itemize} \setbeamertemplate{itemize items}[circle]
% %         \item $\lambda(t) \propto 1/ \PE{\left[\|\nabla \log p_{0|t}(x_t|x_0) \|^2\right]} \Rightarrow$ focus on small $t$~\cite{song2020improved}.
% %         \item $\lambda(t) = 2 \beta(t) \Rightarrow$ 'likelihood weighting'~\cite{song2021maximum,huang2021variational}.
% %     \end{itemize}
% %     \item \textbf{Exponential moving average} of the weights $\theta$ at evaluation due to high stochasticity of the training loss.
% % \end{itemize}
    
% \end{frame}


% \begin{frame}{Important 'tricks' - Learning the score}
% \textbf{Parametrising the score}: Learning the score naively can prove tricky. There is a principled way to parameterise the score however!
% \begin{align*}
%     \rmd \bwd_t = \sbr{-\hlblue{b(T-t, \bwd_t)} + \hlred{\sigma(T-t)}^2 \hlyellow{\v{s}_\theta}(T-t, \bwd_t)}\rmd t +  \hlred{\sigma(T-t)} \rmd \bfB_t
% \end{align*}
% We set
% \begin{equation*}
%     \hlyellow{\v{s}_\theta}(t, \bwd_t) = \v{h}_\theta(t, \bwd_t) / \sigma_t + {2\hlblue{b(t, \bwd_t)}}/{\hlred{\sigma(t)}^2}, \quad \sigma_t = \PE{\left[ \| \nabla \log p(\fwd_t|\fwd_0)\|^2 \right]}^{1/2}.
% \end{equation*}
% \vspace{-\topsep}
% \begin{enumerate}
%     \item This sets the dynamics of the backwards SDE to be the same as the forward if $\v{h}_\theta(t, \bwd_t) = 0$.
%     \item This normalises $\v{h}_\theta(t, \bwd_t)$ to have expected norm of 1.
% \end{enumerate}
% We can show that $\PE{\left[ \| \nabla \log p(\fwd_t|\fwd_0)\|^2 \right]}^{1/2} = \text{Std}[\fwd_t|\fwd_0] = \sigma_t$.
% % This is something we can typically compute for a known SDE.
% %
% % \begin{itemize} \setbeamertemplate{itemize items}[triangle]
% %     \item \textbf{Score network parametrisation}
% %     \begin{itemize} \setbeamertemplate{itemize items}[circle]
% %         \item $\rmd \bwd_t = \hlblue{\left[-b(t, \bwd_t) + 2 \beta(t) \mathbf{s}_\theta(t, \bwd_t) \right]} \rmd t + \hlred{\sqrt{2\beta(t)}} \rmd \bfB_t$.
% %         \item $\hlyellow{\mathbf{s}_\theta}(t, y_t) = \left(h_\theta(t, y_t)/ \sigma_t + 2 * \hlblue{b(t, y_t)}/\beta(t) \right)$.
% %         \item $\rmd \bwd_t = \hlblue{\left[b(t, \bwd_t) + 2 \beta(t) \mathbf{h}_\theta(t, \bwd_t) / \sigma_t \right]} \rmd t + \hlred{\sqrt{2\beta(t)}} \rmd \bfB_t$.
% %         \item If $h_\theta(t, x_t)=0 \Rightarrow \rmd \bwd_t = \rmd \fwd_{t}$.
% %         % \item $\hlyellow{\mathbf{s}_\theta}(t, x_t) = \left(h_\theta(t, x_t)/ \sigma_t + 2*\hlblue{b(t, x_t)}/\beta(t) \right)$.
% %         \item With $\sigma_t \triangleq \PE{\left[ \| \nabla \log p(\fwd_t|\fwd_0)\|^2 \right]}^{1/2} = \text{Std}[\fwd_t|\fwd_0] = 1 - e^{-\int_0^t \beta(s) ds}$.
% %         % \begin{itemize}
% %             \item $1/\sigma_t\rightarrow 0$ as $t\rightarrow \infty$ and $1/\sigma_t \rightarrow \infty$ as $t \rightarrow 0$.
% %         % \end{itemize}
% %         % \item with $\PE{\left[ \| \nabla \log p(\fwd_t|\fwd_0)\|^2 \right]}^{1/2} = \text{Std}[\fwd_t|\fwd_0] \triangleq \sigma_t = 1 - e^{-\int_0^1 \beta(s) ds}$.
% %         % % \item Such that if $h_\theta(t, x_t)=0$, reverse drift $\bar{b}(t, y_t) = -\beta(t) b(t, x_t) + \beta(t) * 2*b(t, x_t)/\beta(t) ={b}(t, y_t)$.
% %         % \item If $h_\theta(t, x_t)=0$, \textbf{forward=backward} since $\hlblue{\bar{b}(t, y_t) = {b}(t, y_t)}$.
% %         % % \item and $\hlblue{b}$ forward drift ($\hlblue{b(t, x_t)}/\beta(t)^2 = -x_t$ if $\M = \rset^d$).
% %         % \item If $\M=\rset$ recovers more or less \cite[][e.g.]{daras2022Soft} [Not exactly as they use $\sigma^2$?].
% %     \end{itemize}
    
% % \end{itemize}
% \end{frame}

% \begin{frame}{Important 'tricks' - Learning the score}
%     The denoising score matching loss has high variance when approximated with Monte Carlo.

%     $\implies$ Using a exponential moving average of the parameters at test time is very impactful.
% \end{frame}




% \section{Likelihood evaluation and connection with continuous normalising flows}

% \begin{frame}{Probability flow}
% \begin{frame}{Likelihood evaluation}
% %
% % \vspace{-0.5em}
% %
% Given the SDE $\rmd \fwd_t = \hlblue{b(t, \fwd_t)} \rmd t + \hlred{\Sigma^{1/2}(t, \fwd_t)} \rmd \bfB_t$, the
% % \begin{equation}
% % \label{eq:}
% % \end{equation}
% \textbf{Fokker-Planck} equation describes the evolution of the density
% \begin{equation}
% \label{eq:fokker_planck}
% \frac{\partial}{\partial t} p_t(x) = -\dive\left( \hlblue{b(t, \cdot)} p_t(\cdot) \right)(x) + \frac{1}{2} \sum_{i,j} \frac{\partial^2}{\partial_i\partial_j} \left( \hlred{\Sigma_{i,j}(t, \cdot)} p_t(\cdot) \right)(x).
% \end{equation}
% \pause
% \vspace{-\topsep}
% \begin{itemize}
%     \setbeamertemplate{itemize items}[triangle]
%     \item If $\hlred{\Sigma = 0}$ (deterministic dynamics): 
% \end{itemize}
% \begin{equation*}
%     \frac{\partial}{\partial t} p_t(x) = -\dive\left( \hlblue{b(t, \cdot)} p_t(\cdot)  \right) (x).
% \end{equation*}
% \pause
% \vspace{-\topsep}
% \begin{itemize} \setbeamertemplate{itemize items}[triangle]
%     \item If $\hlred{\Sigma = \sigma^2(t) \m{I}}$ (Langevin dynamics):
% \end{itemize}
% \vspace{-0.5em}
% \begin{align*}
% \label{eq:fokker_planck_isotropic}
% \frac{\partial}{\partial t} p_t(x) 
% &= -\dive\left( \hlblue{b(t, \cdot)} p_t(\cdot)  \right)(x)  + \tfrac{1}{2} \hlred{\sigma(t)}^2 \Delta p_t (x) \\
% &= -\dive\left( \left[ \hlblue{b(t, \cdot)} - \tfrac{1}{2} \hlred{\sigma(t)}^2 \hlyellow{\nabla \log p_t(\cdot)} \right] p_t(x)  \right).
% \end{align*}
% %
% \end{frame}

% \begin{frame}{Likelihood evaluation (Cont'd)}
% % Since we can express the evolution of the density of the SDE 
% % % $\rmd \fwd_t = \hlblue{b(t, \fwd_t)} \rmd t + \hlred{\Sigma^{1/2}(t, \fwd_t)} \rmd \bfB_t$ 
% % $\rmd \fwd_t = \hlblue{b(t, \fwd_t)} \rmd t + \hlred{\sigma(t)} \rmd \bfB_t$ 
% % in the same fashion as for the deterministic dynamics
% % \begin{equation*}
% %     \frac{\partial}{\partial t} p_t(x) = -\dive\left( \left[ \hlblue{b(t, \cdot)} - \tfrac{1}{2}\hlred{\sigma(t)}^2 \hlyellow{\nabla \log p_t(\cdot)} \right] p_t(\cdot) \right) (x).
% % \end{equation*}
% % \pause
% Then both of the following dynamics
% \begin{enumerate}
%     \item $\rmd \fwd_t = \hlblue{b(t, \fwd_t)} \rmd t + \hlred{\sigma(t)} \rmd \bfB_t$ (stochastic).
%     \item $\rmd \fwd_t = \left[ \hlblue{b(t, \fwd_t)} - \tfrac{1}{2} \hlred{\sigma(t)}^2 \hlyellow{\nabla \log p_t(\fwd_t)} \right] \rmd t$ (deterministic).
% \end{enumerate}
% have the same marginal density $\Pbb_t\triangleq\c{L}(\fwd_t)$ which evolution is given by
% \begin{equation*}
%     \frac{\partial}{\partial t} p_t(x) = -\dive\left( \left[ \hlblue{b(t, \cdot)} - \tfrac{1}{2}\hlred{\sigma(t)}^2 \hlyellow{\nabla \log p_t(\cdot)} \right] p_t(\cdot) \right) (x).
% \end{equation*}
% This gives us a \textit{deterministic} ODE with the same marginal density as the SDE.
% % probability density as the SDE.
% %
% \begin{center}
%     \includegraphics[width=0.8\textwidth]{images/sde.png}
% \end{center}
% %
% \end{frame}

% \begin{frame}{Log-likelihood evolution of ODEs}
% %
% \vspace{-0.5em}
% %
% Assume \textbf{deterministic} evolution of $\fwd_t$ given by the ODE
% \begin{equation}
%     \rmd \fwd_t = \hlblue{b(t, \fwd_t)} \rmd t.
% \end{equation}
% %
% The associated \textbf{Fokker-Planck} equation is
% \begin{equation}
%   \frac{\partial}{\partial t} p_t(\fwd_t) = -\dive\left( \hlblue{b(t, \cdot)} p_t(\cdot) \right) (\fwd_t).
% \end{equation}
% \pause
% %
% The evolution of the log-density is
% \begin{equation}
%   \frac{\partial}{\partial t} \log p_t(x) = -\dive\left( \hlblue{b(t, \cdot)} \right) (x) - \langle \hlblue{b(t, \cdot)}, \nabla \log p_t(x) \rangle.
% \end{equation}
% \pause
% %
% Combining the two dynamics
% \begin{equation}
%   \frac{\rmd}{\rmd t} \log p_t(x) = -\dive\left( \hlblue{b(t, \cdot)} \right) (x).
% \end{equation}
% % \begin{align}
% %     \frac{\rmd}{\rmd t} p_t(\fwd_t) 
% %     &= \frac{\partial}{\partial t} p_t(\fwd_t) + \langle \frac{\partial}{\partial x} p_t(\fwd_t), \frac{\partial}{\partial t} \fwd_t \rangle \\
% %     &= -\dive\left( \hlblue{b(t, \cdot)} p_t \right) (\fwd_t) + \langle \frac{\partial}{\partial x} p_t, \hlblue{b(t, \cdot)} \rangle(\fwd_t) \\
% %      &= -p_t(\fwd_t) ~\dive\left( \hlblue{b(t, \cdot)}\right) (\fwd_t) - \langle \frac{\partial}{\partial x} p_t, \hlblue{b(t, \cdot)} \rangle(\fwd_t) + \langle \frac{\partial}{\partial x} p_t, \hlblue{b(t, \cdot)} \rangle(\fwd_t) \\
% %       \frac{\rmd}{\rmd t} \log p_t(\fwd_t) &= \dive\left( \hlblue{b(t, \cdot)}\right) (\fwd_t).
% % \end{align}
% %
% The \textbf{log-likelihood} can be computed as
% \begin{equation}
%     \log p_0(\fwd_0) = \log p_T(\fwd_T) + \int_0^T  \dive \left( \hlblue{b(t, \cdot)} \right)(\fwd_t) \rmd t.
% \end{equation}

% \end{frame}

% \begin{frame}{Log-likelihood evolution of ODEs}
% %
% \vspace{-0.5em}
% %
% Assume a \textbf{deterministic} evolution of $\fwd_t$ given by the ODE
% \begin{equation}
%     \rmd \fwd_t = \hlblue{b(t, \fwd_t)} \rmd t.
% \end{equation}
% %
% The evolution of the log-density is given by \cite{chen2018neural}
% \begin{equation}
%   \frac{\rmd}{\rmd t} \log p_t(x) = -\dive\left( \hlblue{b(t, \cdot)} \right) (x).
% \end{equation}
% \pause
% % \begin{align}
% %     \frac{\rmd}{\rmd t} p_t(\fwd_t) 
% %     &= \frac{\partial}{\partial t} p_t(\fwd_t) + \langle \frac{\partial}{\partial x} p_t(\fwd_t), \frac{\partial}{\partial t} \fwd_t \rangle \\
% %     &= -\dive\left( \hlblue{b(t, \cdot)} p_t \right) (\fwd_t) + \langle \frac{\partial}{\partial x} p_t, \hlblue{b(t, \cdot)} \rangle(\fwd_t) \\
% %      &= -p_t(\fwd_t) ~\dive\left( \hlblue{b(t, \cdot)}\right) (\fwd_t) - \langle \frac{\partial}{\partial x} p_t, \hlblue{b(t, \cdot)} \rangle(\fwd_t) + \langle \frac{\partial}{\partial x} p_t, \hlblue{b(t, \cdot)} \rangle(\fwd_t) \\
% %       \frac{\rmd}{\rmd t} \log p_t(\fwd_t) &= \dive\left( \hlblue{b(t, \cdot)}\right) (\fwd_t).
% % \end{align}
% %
% Assuming that $\fwd_T \sim p_T$, the \textbf{log-likelihood} can be computed as
% \begin{equation}
%     \log p_0(\fwd_0) = \log p_T(\fwd_T) + \int_0^T  \dive \left( \hlblue{b(t, \cdot)} \right)(\fwd_t) \rmd t.
% \end{equation}
% \end{frame}

% \begin{frame}{Log-likelihood evaluation of ODEs}
% % We can't solve this log-likelihood computation exactly, instead we need to solve the \textbf{augmented ODE}
% The following \textbf{augmented ODE} allows to solve at once the trajectory of $\fwd_t$ and the change in log-likelihood
% \begin{equation}
%     \frac{\rmd}{\rmd t} 
%     \begin{bmatrix} \fwd_t \\ \log p(\fwd_t) \end{bmatrix}
%     = {\begin{bmatrix} \hlblue{b_\theta(t, \cdot)} \\ -\dive\left(\hlblue{b_\theta(t, \cdot)} \right) \end{bmatrix}(\fwd_t)}.
% \end{equation}
% % Which we can do with a myriad of ODE solvers, many of which come with \textit{controllable error}!
% Which can be estimated numerically with a myriad of (adaptive) ODE solvers.
% \pause

% % This is exactly how \textbf{Neural ODEs} \cite{chen2018neural,grathwohl2019Scalable} are trained.
% This is exactly how \textbf{continuous normalising flows} \cite{chen2018neural,grathwohl2019Scalable} are trained.
% Maximising the likelihood ($\c{O}(Nd^2)$ or $\c{O}(Nd)$ with $\dive$ estimator)
% \begin{equation}
%     % {\scriptstyle
%     \PE \left[ \log p_0(\fwd_0) \right] = \PE [  \log p_T(\fwd_T) - \int_0^T \dive( \hlblue{b_\theta(t, \fwd_t)}) \rmd t ].
%     % }
% \end{equation}
% %
% % The 
% % vs \textbf{discrete} normalising flows with invertible map $f_\theta:\rset^d \rightarrow \rset^d$ ($\c{O}(d^3)$)
% % \begin{equation}
% % \PE \left[ \log p_0(\fwd_0) \right] = \PE \left[  \log p_T(f_\theta(\fwd_0)) - \log |\mathrm{D} f_\theta(\fwd_0)|  \right].
% % \end{equation}
% \end{frame}

% \begin{frame}{Probability flow \cite{song2021Scorebased}}
% %
% % We can apply this ODE method directly to our SDEs by converting to the ODE form, and plugging it in! The SDE
% We can apply this likelihood evaluation method for continuous SGMs induced by% the SDE
% \begin{equation}
%   \rmd \fwd_t = \hlblue{\hlblue{b(t, \fwd_t)}} \rmd t + \hlred{\sigma(t)} \rmd \bfB_t
% \end{equation}
% % has the ODE
% since it has the same marginal density has the ODE
% \begin{equation*}
%     \rmd \fwd_t = \left[ \hlblue{b(t, \fwd_t)} - \tfrac{1}{2} \hlred{\sigma(t)}^2 \hlyellow{\nabla \log p_t(\fwd_t)} \right] \rmd t.
% \end{equation*}
% \pause
% % So we can compute the log-likelihood exactly the same way as with the ODEs, by solving the augmented ODE
% We have the associated augmented ODE
% \begin{equation}
%     \frac{\rmd}{\rmd t} 
%     \begin{bmatrix} \fwd_t \\ \log p(\fwd_t) \end{bmatrix}
%     = {\begin{bmatrix} \hlblue{b(t, \fwd_t)} - \tfrac{1}{2} \hlred{\sigma(t)}^2 \hlyellow{\nabla \log p_t(\fwd_t)} \\ 
%     -\dive\left(\hlblue{b(t, \fwd_t)} - \tfrac{1}{2} \hlred{\sigma(t)}^2 \hlyellow{\nabla \log p_t(\fwd_t)}\right) \end{bmatrix}(\fwd_t)}.
% \end{equation}
% We then just have to add on the log likelihood of the reference density, $\log p_T(\fwd_T)$.
% % \begin{equation}
% % %   \rmd \bwd_t = \hlblue{\left[  \bwd_t + 2 \nabla \log p_{T-t}(\bwd_t)\right] } \rmd t + \hlred{\sqrt{2}} \rmd \bfB_t.
% %   \rmd \bwd_t = \hlblue{\left[ -\hlblue{b(T-t, \bwd_t)} + 2 \nabla \log p_{T-t}(\bwd_t)\right] } \rmd t + \hlred{\sqrt{2}} \rmd \bfB_t.
% % \end{equation}
% % \pause
% % %
% % ${\bwd}_t$ has the same distribution as $\hat{\bwd}_t$ with \textbf{determinisitic} dynamics given by the ODE
% % \begin{equation}
% % %   \rmd \hat{\bwd}_t = \hlblue{\left[  \bwd_t + \nabla \log p_{T-t}(\bwd_t)\right] } \rmd t
% %   \rmd \hat{\bwd}_t = \hlblue{\left[ -\hlblue{b(T-t, \bwd_t)} + \nabla \log p_{T-t}(\bwd_t)\right] } \rmd t
% % \end{equation}
% % % since for both ${\bwd}_t$ and $\hat{\bwd}_t$ we have that
% % since they have the same density evolution:
% % \begin{align}
% %   \frac{\partial}{\partial t} \log p_t({\bwd}_t) 
% % %   &= -\dive\left( \hlblue{b(t, \cdot)} p_t \right) ({\bwd}_t) + \tfrac{\hlred{2}}{2} \Delta p_t ({\bwd}_t) \\
% % %   &= -\dive\left( \hlblue{\left[ b(t, \cdot) - \nabla \log p_t \right]} p_t \right) (\hat{\bwd}_t)
% %   &= -\dive\left( \hlblue{\left[-b(T-t, \cdot) + 2 \nabla \log p_{T-t} \right]} p_t \right) ({\bwd}_t) + \tfrac{\hlred{\sqrt{2}}^2}{2} \Delta p_t ({\bwd}_t) \\
% %   &= -\dive\left( \hlblue{\left[ -b(T-t, \cdot) + \nabla \log p_{T-t} \right]} p_t \right) (\hat{\bwd}_t)
% %   = \frac{\partial}{\partial t} \log p_t(\hat{\bwd}_t).
% % \end{align}
% % = 
% % = -\dive\left( \left[ \hlblue{b(t, \cdot)} - \tfrac{\hlred{c}}{2} \nabla \log p_t \right] p_t \right) (x)
%
% \end{frame}


% \begin{frame}{Probability flow \cite{song2021Scorebased}}
% %
% The continuous time reversal process of the forward process
% \begin{equation}
%   \rmd \fwd_t = \hlblue{\hlblue{b(t, \fwd_t)}} \rmd t + \hlred{\sqrt{2}} \rmd \bfB_t
% \end{equation}
% is given by
% \begin{equation}
% %   \rmd \bwd_t = \hlblue{\left[  \bwd_t + 2 \nabla \log p_{T-t}(\bwd_t)\right] } \rmd t + \hlred{\sqrt{2}} \rmd \bfB_t.
%   \rmd \bwd_t = \hlblue{\left[ -\hlblue{b(T-t, \bwd_t)} + 2 \nabla \log p_{T-t}(\bwd_t)\right] } \rmd t + \hlred{\sqrt{2}} \rmd \bfB_t.
% \end{equation}
% \pause
% %
% ${\bwd}_t$ has the same distribution as $\hat{\bwd}_t$ with \textbf{determinisitic} dynamics given by the ODE
% \begin{equation}
% %   \rmd \hat{\bwd}_t = \hlblue{\left[  \bwd_t + \nabla \log p_{T-t}(\bwd_t)\right] } \rmd t
%   \rmd \hat{\bwd}_t = \hlblue{\left[ -\hlblue{b(T-t, \bwd_t)} + \nabla \log p_{T-t}(\bwd_t)\right] } \rmd t
% \end{equation}
% % since for both ${\bwd}_t$ and $\hat{\bwd}_t$ we have that
% since they have the same density evolution:
% \begin{align}
%   \frac{\partial}{\partial t} \log p_t({\bwd}_t) 
% %   &= -\dive\left( \hlblue{b(t, \cdot)} p_t \right) ({\bwd}_t) + \tfrac{\hlred{2}}{2} \Delta p_t ({\bwd}_t) \\
% %   &= -\dive\left( \hlblue{\left[ b(t, \cdot) - \nabla \log p_t \right]} p_t \right) (\hat{\bwd}_t)
%   &= -\dive\left( \hlblue{\left[-b(T-t, \cdot) + 2 \nabla \log p_{T-t} \right]} p_t \right) ({\bwd}_t) + \tfrac{\hlred{\sqrt{2}}^2}{2} \Delta p_t ({\bwd}_t) \\
%   &= -\dive\left( \hlblue{\left[ -b(T-t, \cdot) + \nabla \log p_{T-t} \right]} p_t \right) (\hat{\bwd}_t)
%   = \frac{\partial}{\partial t} \log p_t(\hat{\bwd}_t).
% \end{align}
% % = 
% % = -\dive\left( \left[ \hlblue{b(t, \cdot)} - \tfrac{\hlred{c}}{2} \nabla \log p_t \right] p_t \right) (x)

% \end{frame}


% \begin{frame}{Continuous normalising flows (CNFs) ~\cite{chen2018neural,grathwohl2019Scalable}}
% %
% Assume \textbf{deterministic} forward evolution of $\fwd_t$ with dynamics given by the ODE
% \begin{equation}
%     \rmd \fwd_t = \hlblue{b_\theta(t, \fwd_t)} \rmd t
%     \quad \text{thus a backward evolution} \quad \rmd \bwd_t = \hlblue{b_\theta(T - t, \bwd_t)} \rmd t
% \end{equation}
% where $\hlblue{b_\theta}: \rset_+ \times \rset^d \rightarrow \rset^d$ is a parametric family of drifts (i.e.\ vector fields). 
% \pause
% %

% Consider the \textbf{augmented} ODE
% \begin{equation}
%     \frac{\rmd}{\rmd t} 
%     \begin{bmatrix} \fwd_t \\ \log p(\fwd_t) \end{bmatrix}
%     = \hlblue{\begin{bmatrix} b_\theta(t, \cdot) \\ -\dive\left(b_\theta(t, \cdot) \right) \end{bmatrix}(\fwd_t)}.
% \end{equation}
% \pause
% %
% Train drift \hlblue{b_\theta} by maximising the likelihood ($\c{O}(Nd^2)$ or $\c{O}(Nd)$ with $\dive$ estimator)
% \begin{equation}
%     % {\scriptstyle
%     \PE \left[ \log p_0(\fwd_0) \right] = \PE [  \log p_T(\fwd_T) - \int_0^T \dive( \hlblue{b_\theta(s, \fwd_s)}) \rmd s ].
%     % }
% \end{equation}
% %
% % The 
% vs \textbf{discrete} normalising flows with invertible map $f_\theta:\rset^d \rightarrow \rset^d$ ($\c{O}(d^3)$)
% \begin{equation}
% \PE \left[ \log p_0(\fwd_0) \right] = \PE \left[  \log p_T(f_\theta(\fwd_0)) - \log |\mathrm{D} f_\theta(\fwd_0)|  \right].
% \end{equation}
% \end{frame}


\begin{frame}{Recap: Continuous diffusion models}
 %
 \begin{center}
        \includegraphics[width=\textwidth]{images/sde.png}
    \end{center}
\begin{itemize} \setbeamertemplate{itemize items}[triangle]
    \item Continuously \textbf{noise} data samples with forward SDE
    \item Aim: time-reversal of this process $\Rightarrow$ \textbf{denoising} process
    % \begin{itemize} \setbeamertemplate{itemize items}[circle]
    % \item Same variance as the forward process
    % \item Its mean depends on the forward process's mean and the \textbf{Stein score}
    % \item The score is parametrised and trained by learning to `denoise' samples
    % \item Generate samples by discretising the (approximate) backward process with Euler-Maruyama
    % \end{itemize}
\end{itemize}
%
% \begin{itemize}
%     \item Idea: Use a \textit{continuous} series of noise scales!
%     \item Do this by constructing an \textbf{SDE} forward noising process $(\fwd_t)_{t \in \ccint{0,T}}$.
%     \item Have this noising converge to a \textbf{known distribution}.
%     \item \textbf{Invert} this SDE noising process to get $(\bwd_t)_{t\in[0,T]} = (\fwd_{T-t})_{t}$.
% \end{itemize}
\end{frame}

% \section{A variational perspective}

% % \begin{frame}{Equivalence between SM and ELBO loss}
% % \begin{frame}{Discrete SGMs}

% % \todo[inline]{To fill}
% % \cite{sohl2015deep,ho2020denoising,nichol2021improved}
% % \end{frame}


% \begin{frame}{Discrete SGMs: ELBO \cite{sohl2015deep,ho2020denoising}}
% %
% We assume a discrete process induced by the \textbf{forward transition} $q(\v{y}_{t}|\v{y}_{t-1})$:
%     % $p_{k+1|k}(\v{y}_{k+1}|\v{y}_{k})$ (e.g.\ $\c{N}(\mu_k(\v{y}_k), \sigma_k)$
%      % (e.g.\ $\c{N}(\mu_k(\v{y}_t), \sigma_t)$)
% %
%     \begin{equation}\label{eq:forward_markov}
%          q(\v{y}_{0:T}) = q(\v{y}_0) \prod_{t=1}^{T} q(\v{y}_{t}|\v{y}_{t-1}).
%     \end{equation}
% %
% The (intractable) \textbf{backward transition} $p_{t-1|t}(\v{y}_{t-1}|\v{y}_{t})$ 
% % $p_{t|t+1}(\v{y}_{t}|\v{y}_{t+1}) = p_{t+1|t}(\v{y}_{t+1}|\v{y}_{t}) p_{t}(\v{y}_{t})/ p_{t+1}(\v{y}_{t+1})$ 
% induces a backward process% given by the Markov chain
% % For Gaussian forward process, the \textbf{backwards process}, $p(x_{t-1}|x_{t})$, is also (approximately, for sufficiently small steps) Gaussian with intractable mean and closed-form variance, yielding
%     % \begin{equation}\label{eq:backward_markov}
%     %     p(x_{0:N}) = p(x_N) \prod_{k=0}^{N-1} p_{k|k+1}(x_{k}|x_{k+1}).
%     % \end{equation}
%     \begin{equation}\label{eq:backward_markov}
%         p({\v{y}}_{0:T}) = p({\v{y}}_T) \prod_{t=T}^{1} p({\v{y}}_{t-1}|{\v{y}}_{t}).%, \quad p({x}_{t-1}|{x}_{t}) \approx \c{N}\del{{x}_{t-1}\middle| \v{\mu}\del{{x}_t, t}, \m{\Sigma}_t}.
%     \end{equation}
%     \pause
%     %
%     % But how do we \textbf{train} an approximate mean function ${\v{\mu}_\theta}\del{\v{y}_t, t}$?
% A bound on the (negative) \textbf{log-likelihood} can be derived as
%     \begin{align*}
%         \E\sbr{-\log p_\theta(\v{y}_0)} \leq \E_q\sbr{- \log \frac{p_\theta(\v{y}_{0:T})}{q(\v{y}_{1:T}|\v{y}_0)}} = \E_q \sbr{ -\log p(\v{y}_T) - \sum_{t \geq 1}\log \frac{p_\theta(\v{y}_{t-1}| \v{y}_{t})}{q(\v{y}_t | \v{y}_{t-1})}}
%         \triangleq \c{E}
%         % \log p_\theta({x_0}) &\geq \E_{q({x}_{1:T}|{x}_0)}\sbr{\log \frac{p_\theta({x}_{0:T})}{q({x}_{1:T}|{x}_0)}} = \E_{q({x}_{1:T}|{x}_0)} \sbr{\log p({x}_T) + \sum_{t \geq 1}\log \frac{p_\theta({x}_{t-1}| {x}_{t})}{q({x}_t | {x}_{t-1})}} 
%         % \\ &\geq
%     \end{align*}
%     which can straightforwardly be estimated via Monte Carlo sampling.
%     % All these terms are computable, and so we could train via Monte Carlo sampling and SGD.

% \end{frame}


% \begin{frame}{A more efficient EBLO}
% %
% \vspace{-1.2em}
% \begin{equation*}
%     \c{E} = 
%     \E_q \sbr{
%         \underbrace{
%             D_{KL} \del{q(\v{y}_T | \v{y}_0) \middle| \middle| p(\v{y}_T)}
%         }_{\hlblue{\scriptstyle L_T}}
%          + \sum_{t>1} \underbrace{
%             D_{KL}\del{q(\v{y}_{t-1}| \v{y}_t, \v{y}_0)  \middle| \middle | p_\theta(\v{y}_{t-1} | \v{y}_t)}
%         }_{\hlred{\scriptstyle L_{t-1}}}
%         \underbrace{- \log p_\theta(\v{y}_0 | \v{y}_1)}_{\hlorange{\scriptstyle L_0}}
%     }.
% \end{equation*}
% \vspace{-1.0em}
% %
% \begin{itemize}
%     \item[\hlblue{L_T}] \textit{prior matching}: constant if we fix $q$
%     \item[\hlorange{L_0}] \textit{reconstruction}: can compute directly
%     \item[\hlred{L_{t-1}}] \textit{denoising matching}: requires access to $q(\v{y}_{t-1}| \v{y}_t, \v{y}_0)=q(\v{y}_{t-1}| \v{y}_0)$
% \end{itemize}
% \pause
% % These are all closed form KL divergences!
% % Fortunately we can get at this! $q(\v{y}_{t-1}| \v{y}_t, \v{y}_0) = \c{N}\del{\v{y}_{t-1} \middle | \tilde{\v{\mu}}_t\del{\v{y}_t, \v{y}_0}, \tilde{\beta}_t \m{I}}$ with
% % \begin{equation*}
% %     \tilde{\v{\mu}}_t\del{\v{y}_t, \v{y}_0} = \frac{\sqrt{1-\alpha_{t-1}} \beta_t}{\alpha_t}\v{y}_0 + \frac{\sqrt{1-\alpha_t}(\alpha_{t-1})}{\alpha_t}\v{y}_t \; \text{and} \; \tilde{\beta}_t = \frac{\alpha_{t-1}}{\alpha_t}\beta_t
% % \end{equation*}
% Denoting $\v{\mu}_q\del{x_t, x_0} = \PE[\fwd_t|\fwd_0]$ and ${\sigma}_q\del{t} = \text{Std}[\fwd_t|\fwd_0]$

% Choosing $p(\v{y}_{t-1}|\v{y}_{t}) \approx p_\theta(\v{y}_{t-1}|\v{y}_{t}) \triangleq \c{N}\del{\v{y}_{t-1}\middle| \v{\mu}_\theta \del{t, \v{y}_t}, \sigma^2_q(t) \Id}$ we have
% %
% \begin{align*}
%  \Rightarrow \arg\min_\theta D_{KL}\del{q(\v{y}_{t-1}| \v{y}_t, \v{y}_0)  \middle| \middle | p_\theta(\v{y}_{t-1} | \v{y}_t)} 
%  &= \arg\min_\theta \tfrac{1}{2}  \tfrac{1}{\sigma^2_q\del{t}} \left[\| \v{\mu}_\theta\del{t, \v{y}_t} - \v{\mu}_q\del{\v{y}_t, \v{y}_0} \|^2\right] \\
%  &\hspace{-5em} = \arg\min_\theta  \tfrac{1}{2}\tfrac{1}{ \sigma^2_q\del{t}} C(t) \left[\| \hlyellow{\mathbf{s}_\theta}\del{t, \v{y}_t} - \nabla \log q\del{\v{y}_t|\v{y}_0} \|^2\right].
% \end{align*}
% %
% \end{frame}



% % \begin{frame}{Continuous SGMs: Bounding KL divergence}
% \begin{frame}{Continuous SGMs: Score matching as ELBO}
% % \cite{huang2021variational,song2021maximum}
% %
% We now assume a continuous process $\rmd \fwd_t = \hlblue{b(t, \fwd_t)} \rmd t + \hlred{\sigma(t)} \rmd \bfB_t$ with reversal
% \begin{equation}
% % \rmd \fwd_t = \hlblue{b(t, \fwd_t)} \rmd t + \hlred{\sigma(t)} \rmd \bfB_t.
% \rmd \bwd_t = \left[ -\hlblue{b(T-t,\bwd_t)} + \hlred{\sigma^2(T-t)} \hlyellow{\nabla \log p_{T-t}(\bwd_t)}\right]  \rmd t + \hlred{\sigma(T-t)} \rmd \bfB_t.
% \end{equation}
% \pause
% %
% % We remind the score matching losses:
% % \begin{align*}
% %     \ell_{\mathrm{sm}}(\hlyellow{\mathbf{s}_\theta}; \lambda(\cdot)) &\triangleq \tfrac{1}{2}\PE_{t,x_t}{\left[ \lambda(t) \|\hlyellow{\mathbf{s}_\theta}(t, x_t) - \nabla_{x_t} \log p_{t}(x_t)\|^2 \right]} \\
% %     \ell_{\mathrm{dsm}}(\hlyellow{\mathbf{s}_\theta}; \lambda(\cdot)) &\triangleq \tfrac{1}{2}\PE_{x_0,t,x_t}{\left[ \lambda(t) \|\hlyellow{\mathbf{s}_\theta}(t, x_t) - \nabla_{x_t} \log p_{t|0}(x_t|x_0)\|^2 \right]}\\
% %     % \ell_{\mathrm{ism}}(\hlyellow{\mathbf{s}_\theta}; \lambda(\cdot)) &\triangleq \PE_{x_0,t,x_t}{\left[ \lambda(t) \left(\tfrac{1}{2} \|\hlyellow{\mathbf{s}_\theta}(t, x_t)\|^2 + \dive \left(\hlyellow{\mathbf{s}_\theta}(t, \cdot)\right)(x_t) \right) \right]}
% % \end{align*}
% % \pause
% % \vspace{-3em}
% %
% \begin{theorem}{\cite{song2021maximum}}{}
% Under some regularity assumptions, setting the weighting to $\lambda(t) = \sigma^2(t)$:
% \begin{equation}
% \mathrm{D}_{\text{KL}}(p | p_\theta) 
% \le \ell_{\mathrm{sm}}(\hlyellow{\mathbf{s}_\theta}; \sigma^2(\cdot)) + \mathrm{D}_{\text{KL}}(p_T | \piinv)
% = \ell_{\mathrm{dsm}}(\hlyellow{\mathbf{s}_\theta}; \sigma^2(\cdot)) + C + \mathrm{D}_{\text{KL}}(p_T | \piinv).
% \end{equation}
% \begin{align*}
% \E_{p_0(\v{y})}\sbr{-\log p_\theta(\v{y})} \leq
% % \mathrm{D}_{\text{KL}}(p | p_\theta)  
% \ell_{\mathrm{sm}}(\hlyellow{\mathbf{s}_\theta}; \sigma^2(\cdot)) + C_1
% = \ell_{\mathrm{dsm}}(\hlyellow{\mathbf{s}_\theta}; \sigma^2(\cdot)) + C_2.
% \end{align*}
% \end{theorem}

% % You get this by seeing, for path measures $\v{\mu}$ and $\v{\nu}$ on the backwards and approx. SDEs.
% % \begin{equation*}
    
% % \end{equation*}
% \end{frame}

% \begin{frame}{ELBOs for single datapoints}
% \begin{theorem}{\cite{song2021maximum,huang2021variational}}{}
% \begin{equation*}
%     -\log p_\theta(\v{y}) \leq \c{L}_\theta^{SM}(\v{y}) = \c{L}_\theta^{DSM}(\v{y})    
% \end{equation*}
% \end{theorem}
% \hfill
% \begin{align*}
%     &\c{L}_\theta^{SM}(\v{y}_0) = -\overbrace{\E_{p(\v{y}_T | \v{y}_0)} \sbr{\log p_{ref}(\v{y}_T)}}^{\text{const.}}\\ 
%     &\quad+\frac{1}{2} \int_0^T \E_{p(\v{y}_t | \v{y}_0)}\sbr{ \underbrace{2\hlred{\sigma(t)}^2\nabla_{\v{y}_t} \cdot \hlyellow{s_\theta}(t, \v{y}_t) + \hlred{\sigma(t)}^2 \norm{\hlyellow{s_\theta}(t, \v{y}_t)}^2}_{\text{implicit score matching}} - \underbrace{2\nabla_{\v{y}_t} \cdot \hlblue{b(t, \v{y}_t)}}_{\text{const.}}  } \rmd t\\
% \end{align*}
% \hfill
% \end{frame}

% \begin{frame}{ELBOs for single datapoints}
% \begin{theorem}{\cite{song2021maximum,huang2021variational}}{}
% \begin{equation*}
%     -\log p_\theta(\v{y}) \leq \c{L}_\theta^{SM}(\v{y}) = \c{L}_\theta^{DSM}(\v{y})    
% \end{equation*}
% \end{theorem}
% \vspace{-0.5em}
% \begin{align*}
%     &\c{L}_\theta^{DSM}(\v{y}_0) =  -\overbrace{\E_{p(\v{y}_T | \v{y}_0)} \sbr{\log p_{ref}(\v{y}_T)}}^{\text{const.}}\\ 
%     &+\quad\frac{1}{2} \int_0^T \E_{p(\v{y}_t | \v{y}_0)}\sbr{ \underbrace{\hlred{\sigma(t)}^2 \norm{ \hlyellow{s_\theta}(t, \v{y}_t) - \nabla_{\v{y}_t} \log p(\v{y}_t | \v{y}_0) }^2}_{\text{denoising score matching}}} \rmd t \\
%     &-\quad\frac{1}{2} \int_0^T \E_{p(\v{y}_t | \v{y}_0)}\sbr{\underbrace{\hlred{\sigma(t)}^2 \norm{\nabla_{\v{y}_t} \log p(\v{y}_t | \v{y}_0) }^2 + 2\nabla_{\v{y}_t} \cdot \hlblue{b(t, \v{y}_t)}}_{\text{const.}} } \rmd t
% \end{align*}
% \end{frame}

% \begin{frame}{Continuous-time ELBO}
% %
% \vspace{-.3em}
% \begin{theorem}{\cite{huang2021variational,song2021maximum}}{}
% %
% % \begin{equation}
% % \c{E}^\infty \triangleq  \PE{\left[ -\tfrac{1}{2} \int_0^T  \| \mathbf{s}_\theta(t, \omega) \|^2_2 \rmd t + \log p_0(\bwd_T) - \int_0^T \nabla \cdot b(t, ??) ~\rmd t | \bwd_0=x \right]}
% % \end{equation}
% % where the expectation is taken w.r.t.\ the Brownian motion $\hat{B}$ ???, and $\bwd_t$ solves $\rmd \bwd_t = [- b(T-t, \cdot) + \sigma(T-t) \mathbf{s}_\theta](\bwd_t) \rmd t + \sigma(T-t) \rmd B_t$.
% %
% Under some regularity assumptions, a continuous-time ELBO $\c{E}^\infty$ lower bounding the model log-likelihood $\log p_\theta(x)$ can be written as (Feynman-Kac)
% \begin{align*}
% % \c{E}^\infty \triangleq  \PE{\left[ -\tfrac{1}{2} \int_0^T  \| \mathbf{s}_\theta(t, \omega) \|^2_2 \rmd t + \log p_0(\bwd_T) - \int_0^T \nabla \cdot b(t, ??) ~\rmd t \right]}
% % \log p_\theta(x) \le 
% \c{E}^\infty 
% &\triangleq \PE_{\mathbb{Q}}{\left[ \log \frac{\rmd \mathbb{P}}{\rmd \mathbb{Q}} + \log \piinv(\fwd_T) - \int_0^T \dive (\sigma^2 \hlyellow{\mathbf{s}_\theta}- b)(\fwd_t) \rmd t ~\middle|~ \fwd_0=x \right]} \\
% &= \PE_{p_{T|0}(x_T|x)}{\left[ \log \piinv(x_T) \right]}
% - \int_0^T \PE_{p_{t|0}(x_t|x)}{\left[ \tfrac{1}{2} \sigma^2 \| \hlyellow{\mathbf{s}_\theta} \|^2_2 + \dive (\sigma^2 \hlyellow{\mathbf{s}_\theta}- b) \right]} \rmd t.
% \end{align*}
% The variational gap can be written as the (explicit) score matching loss:
% \begin{equation}
% % \log p_\theta(x) - \c{E}^\infty = \int_0^T \PE{\left[ \sigma^2(t) \| \mathbf{s}_\theta(t, \omega) - \nabla \log p_{T-t}(\bwd_t) \|^2_2 \right]} \rmd t.
% \log p_\theta(x) - \c{E}^\infty = \int_0^T \PE{\left[ \sigma^2(t) \| \hlyellow{\mathbf{s}_\theta}(t, x_t) - \nabla \log p_{t}(x_t) \|^2_2 \right]} \rmd t.
% \end{equation}
% In particular it is tight, i.e.\ $\log p_\theta(x) = \c{E}^\infty$, if $\hlyellow{\mathbf{s}_\theta}(t, x) = \nabla \log p_{t}(x)$ a.e.\ .

% %
% % Under some regularity assumptions, the log-likelihood can be bounded as
% % \begin{align*}
% % \log p_\theta(x) &\le \PE_{p_{T|0}(x_T|x)}{\left[ \log \piinv(x_T) \right]} 
% % - \int_0^T \PE_{p_{t|0}(x_t|x)}{\left[ \nabla \cdot b(t, x_t) \right]} \rmd t \\
% % &-\tfrac{1}{2} \int_0^T \PE_{p_{t|0}(x_t|x)}{\left[ \sigma^2(t) \| \mathbf{s}_\theta(t, \omega) - \nabla \log p_{t|0}(x_t|x_0) \|^2_2 \right]} \rmd t.
% % \end{align*}
% % This bound is tight if $\mathbf{s}_\theta(t, x) = \nabla \log p_{T-t}(x)$ almost everywhere.
% %
% \end{theorem}
% \end{frame}

% \section{Active research directions}

% \begin{frame}{Active research directions}

% \begin{itemize}  \setbeamertemplate{itemize items}[triangle]
%     \item \textit{Accelerate} reverse sampling
%     \item \textit{Structured data}: text, graph, discrete, manifold, protein, functions etc
%     \item \textit{Latent} manipulation
%     \item \textit{Scaling} for large models
%     % \item Connections with other methods
%     \item \textit{Theory}: how come it works so well?
%     % \item ??
% \end{itemize}
% \end{frame}
